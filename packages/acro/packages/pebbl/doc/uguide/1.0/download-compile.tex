\section{Downloading and Compiling PEBBL}


The PEBBL software project is supported by the Acro optimization
project.  Acro supports the integration of PEBBL builds with auxillary
libraries, like UTILIB.


\subsection{Downloading PEBBL}

The Acro software library can be installed from a distribution file
(tape, etc.) or using a checkout from the Concurrent Version System (CVS)
repository. If you are accessing current files from the CVS repository,
you need to use the {\bf cvs.a} and {\bf ssh.cvs} scripts, which can be
downloaded from http://software.sandia.gov/Acro.  The Acro library can
be checked out of the cvs repository by executing:
\begin{verbatim}
   cvs.a checkout acro-pebbl
\end{verbatim}
The use of {\bf cvs.a} requires an account on the machine software.sandia.gov,
which is generally only available to Sandians and academic collaborators
of Sandians.

The latest version of PEBBL can also be acquired in compressed tarball
form from 
\begin{verbatim}
   http://software/sandia.gov/Acro
\end{verbatim}
Once downloaded, compressed tarballs
for the version of the day (VOTD), for example, 
can be downloaded and extracted with the 
following:
\begin{verbatim}
   gzip -d pebbl-VOTD.tar.gz | tar xf -
\end{verbatim}


\subsection{Configuring and Building}

PEBBL can be configured using the standard build syntax:
\begin{verbatim}
   cd acro
   ./configure
   make
\end{verbatim}
This builds the library libpebbl.a in acro/lib, along with supporting
libraries. The library headers are installed in acro/include.
PEBBL developers using CVS need to build with the autoconf tools.  This can be
done with the following syntax:
\begin{verbatim}
   cd acro
   autoreconf
   ./configure
   make
\end{verbatim}
The acro/setup command can also be used to simplify this process:
\begin{verbatim}
   cd acro
   ./setup configure make
\end{verbatim}
The {\bf setup} command generates the following files in the acro/test
directory:
\begin{verbatim}
   config.out             The output of `autoreconf' and `configure'
   config.xml             Summary of config.out to detect errors
   build.out              The output of `make'
   build.xml              Summary of build.out to detect errors
\end{verbatim}

The Acro build can be modified through the use of environment variables
and with command line options to the {\bf configure} script.  If you set no
environment variables and use no command line options, the {\bf configure}
script will choose sensible values.
\begin{verbatim}
   configure --help
\end{verbatim}
This will list all the configuration options, and in some instances the
default action.  Typical use might be to define where MPI is located,
and in which directory to install the Acro libraries and header files:
\begin{verbatim}
   configure --with-mpi-compilers=/usr/local/mpich-1.2.4/ch_p4/bin
          --prefix=/Net/usr/local
\end{verbatim}
A detailed discussion of these configuration options is given in the file
\begin{verbatim}
   acro/INSTALL
\end{verbatim}
and on the Acro web pages (see http://software/sandia.gov/Acro).
