\chapter{Reliability Methods}\label{uq:reliability}

This chapter explores local and global reliability methods in greater
detail than that provided in the Uncertainty Quantification chapter of
the User's Manual.


\section{Local Reliability Methods}\label{uq:reliability:local}

Local reliability methods include the Mean Value method and the family
of most probable point (MPP) search methods.


\subsection{Mean Value}\label{uq:reliability:local:mv}

The Mean Value method (MV, also known as MVFOSM in \cite{Hal00}) is
the simplest, least-expensive reliability method because it estimates
the response means, response standard deviations, and all CDF/CCDF
response-probability-reliability levels from a single evaluation of
response functions and their gradients at the uncertain variable
means.  This approximation can have acceptable accuracy when the
response functions are nearly linear and their distributions are
approximately Gaussian, but can have poor accuracy in other
situations.  The expressions for approximate response mean $\mu_g$,
approximate response variance $\sigma^2_g$, response target to
approximate probability/reliability level mapping ($\bar{z} \to p,\beta$),
and probability/reliability target to approximate response level mapping
($\bar{p},\bar{\beta} \to z$) are

\begin{eqnarray}
\mu_g      & = & g(\mu_{\bf x})  \label{eq:mv_mean1} \\
\sigma^2_g & = & \sum_i \sum_j Cov(i,j) \frac{dg}{dx_i}(\mu_{\bf x})
                 \frac{dg}{dx_j}(\mu_{\bf x}) \label{eq:mv_std_dev} \\
\beta_{cdf}  & = & \frac{\mu_g - \bar{z}}{\sigma_g} \label{eq:mv_ria_cdf} \\
\beta_{ccdf} & = & \frac{\bar{z} - \mu_g}{\sigma_g} \label{eq:mv_ria_ccdf} \\
z        & = & \mu_g - \sigma_g \bar{\beta}_{cdf} \label{eq:mv_pma_cdf} \\
z        & = & \mu_g + \sigma_g \bar{\beta}_{ccdf} \label{eq:mv_pma_ccdf}
\end{eqnarray}

respectively, where ${\bf x}$ are the uncertain values in the 
space of the original uncertain variables (``x-space''), $g({\bf x})$
is the limit state function (the response function for which
probability-response level pairs are needed), and $\beta_{cdf}$ and
$\beta_{ccdf}$ are the CDF and CCDF reliability indices, respectively.

With the introduction of second-order limit state information, MVSOSM
calculates a second-order mean as

\begin{equation}
\mu_g = g(\mu_{\bf x}) + \frac{1}{2} \sum_i \sum_j Cov(i,j) 
\frac{d^2g}{dx_i dx_j}(\mu_{\bf x}) \label{eq:mv_mean2}
\end{equation}

This is commonly combined with a first-order variance
(Equation~\ref{eq:mv_std_dev}), since second-order variance involves
higher order distribution moments (skewness, kurtosis)~\cite{Hal00}
which are often unavailable.

The first-order CDF probability $p(g \le z)$, first-order 
CCDF probability $p(g > z)$, $\beta_{cdf}$, and $\beta_{ccdf}$ are
related to one another through

\begin{eqnarray}
p(g \le z)   & = & \Phi(-\beta_{cdf})     \label{eq:p_cdf} \\
p(g > z)     & = & \Phi(-\beta_{ccdf})    \label{eq:p_ccdf} \\
\beta_{cdf}  & = & -\Phi^{-1}(p(g \le z)) \label{eq:beta_cdf} \\
\beta_{ccdf} & = & -\Phi^{-1}(p(g > z))   \label{eq:beta_ccdf} \\
\beta_{cdf}  & = & -\beta_{ccdf}          \label{eq:beta_cdf_ccdf} \\
p(g \le z)   & = & 1 - p(g > z)           \label{eq:p_cdf_ccdf}
\end{eqnarray}

where $\Phi()$ is the standard normal cumulative distribution
function.  A common convention in the literature is to define $g$ in
such a way that the CDF probability for a response level $z$ of zero
(i.e., $p(g \le 0)$) is the response metric of interest.  DAKOTA is
not restricted to this convention and is designed to support CDF or
CCDF mappings for general response, probability, and reliability level
sequences.

With the Mean Value method, it is possible to obtain 
importance factors indicating the relative importance of 
input variables.  The importance factors can be viewed
as an extension of linear sensitivity analysis combining deterministic
gradient information with input uncertainty information,
\emph{i.e}. input variable standard deviations. The accuracy of the
importance factors is contingent of the validity of the linear
approximation used to approximate the true response functions.
The importance factors are determined as: 

\begin{equation}
ImpFactor_i  = ({\frac{\sigma_{x_{i}}}{\sigma_g}}{\frac{dg}{dx_i}(\mu_{\bf x})})^2
\end{equation}


\subsection{MPP Search Methods}\label{uq:reliability:local:mpp}

All other local reliability methods solve an equality-constrained nonlinear
optimization problem to compute a most probable point (MPP) and then
integrate about this point to compute probabilities.  The MPP search
is performed in uncorrelated standard normal space (``u-space'') since
it simplifies the probability integration: the distance of the MPP
from the origin has the meaning of the number of input standard
deviations separating the mean response from a particular response
threshold.  The transformation from correlated non-normal
distributions (x-space) to uncorrelated standard normal distributions
(u-space) is denoted as ${\bf u} = T({\bf x})$ with the reverse
transformation denoted as ${\bf x} = T^{-1}({\bf u})$.  These
transformations are nonlinear in general, and possible approaches
include the Rosenblatt~\cite{Ros52}, Nataf~\cite{Der86}, and
Box-Cox~\cite{Box64} transformations.  The nonlinear transformations
may also be linearized, and common approaches for this include the
Rackwitz-Fiessler~\cite{Rac78} two-parameter equivalent normal and the
Chen-Lind~\cite{Che83} and Wu-Wirsching~\cite{Wu87} three-parameter
equivalent normals.  DAKOTA employs the Nataf nonlinear transformation
which is suitable for the common case when marginal distributions and
a correlation matrix are provided, but full joint distributions are
not known\footnote{If joint distributions are known, then the
Rosenblatt transformation is preferred.}.  This transformation occurs 
in the following two steps.  To transform between the
original correlated x-space variables and correlated standard normals
(``z-space''), a CDF matching condition is applied for each of the
marginal distributions:
\begin{equation}
\Phi(z_i) = F(x_i) \label{eq:trans_zx}
\end{equation}
where $F()$ is the cumulative distribution function of the original
probability distribution.  Then, to transform between correlated
z-space variables and uncorrelated u-space variables, the Cholesky 
factor ${\bf L}$ of a modified correlation matrix is used:
\begin{equation}
{\bf z} = {\bf L} {\bf u} \label{eq:trans_zu}
\end{equation}
where the original correlation matrix for non-normals in x-space has
been modified to represent the corresponding ``warped'' correlation in 
z-space~\cite{Der86}.

The forward reliability analysis algorithm of computing CDF/CCDF
probability/reliability levels for specified response levels is called
the reliability index approach (RIA), and the inverse reliability
analysis algorithm of computing response levels for specified CDF/CCDF
probability/reliability levels is called the performance measure
approach (PMA)~\cite{Tu99}.  The differences between the RIA and PMA
formulations appear in the objective function and equality constraint
formulations used in the MPP searches.  For RIA, the MPP search for
achieving the specified response level $\bar{z}$ is formulated as
computing the minimum distance in u-space from the origin to the
$\bar{z}$ contour of the limit state response function:
\begin{eqnarray}
{\rm minimize}     & {\bf u}^T {\bf u} \nonumber \\
{\rm subject \ to} & G({\bf u}) = \bar{z} \label{eq:ria_opt}
\end{eqnarray}

and for PMA, the MPP search for achieving the specified
reliability/probability level $\bar{\beta},\bar{p}$ is formulated as
computing the minimum/maximum response function value corresponding
to a prescribed distance from the origin in u-space:
\begin{eqnarray}
{\rm minimize}     & \pm G({\bf u}) \nonumber \\
{\rm subject \ to} & {\bf u}^T {\bf u} = \bar{\beta}^2 \label{eq:pma_opt}
\end{eqnarray}

where ${\bf u}$ is a vector centered at the origin in 
u-space and $g({\bf x}) \equiv G({\bf u})$ by definition.  In the RIA
case, the optimal MPP solution ${\bf u}^*$ defines the reliability 
index from $\beta = \pm \|{\bf u}^*\|_2$, which in turn defines the 
CDF/CCDF probabilities (using Equations~\ref{eq:p_cdf}-\ref{eq:p_ccdf} in 
the case of first-order integration).  The sign of $\beta$ is defined by
\begin{eqnarray}
G({\bf u}^*) > G({\bf 0}): \beta_{cdf} < 0, \beta_{ccdf} > 0 \\
G({\bf u}^*) < G({\bf 0}): \beta_{cdf} > 0, \beta_{ccdf} < 0
\end{eqnarray}
\noindent where $G({\bf 0})$ is the median limit state response computed 
at the origin in u-space\footnote{It is not necessary to explicitly compute
the median response since the sign of the inner product 
$\langle {\bf u}^*, \nabla_{\bf u} G \rangle$
can be used to determine the orientation of the optimal response with 
respect to the median response.} (where $\beta_{cdf}$ = $\beta_{ccdf}$ = 0 
and first-order $p(g \le z)$ = $p(g > z)$ = 0.5).  In the PMA case, the 
sign applied to $G({\bf u})$ (equivalent to minimizing or maximizing 
$G({\bf u})$) is similarly defined by $\bar{\beta}$
\begin{eqnarray}
\bar{\beta}_{cdf} < 0, \bar{\beta}_{ccdf} > 0: {\rm maximize \ } G({\bf u}) \\
\bar{\beta}_{cdf} > 0, \bar{\beta}_{ccdf} < 0: {\rm minimize \ } G({\bf u})
\end{eqnarray}
and the limit state at the MPP ($G({\bf u}^*)$) defines the desired
response level result.

\subsubsection{Limit state approximations} \label{uq:reliability:local:mpp:approx}

There are a variety of algorithmic variations that are available for
use within RIA/PMA reliability analyses.  First, one may select among
several different limit state approximations that can be used to
reduce computational expense during the MPP searches.  Local,
multipoint, and global approximations of the limit state are possible.
\cite{Eld05} investigated local first-order limit state 
approximations, and \cite{Eld06a} investigated local second-order
and multipoint approximations.  These techniques include:

\begin{enumerate}
\item a single Taylor series per response/reliability/probability level 
in x-space centered at the uncertain variable means.  The first-order 
approach is commonly known as the Advanced Mean Value (AMV) method:
\begin{equation}
g({\bf x}) \cong g(\mu_{\bf x}) + \nabla_x g(\mu_{\bf x})^T 
({\bf x} - \mu_{\bf x}) \label{eq:linear_x_mean}
\end{equation}
and the second-order approach has been named AMV$^2$:
\begin{equation}
g({\bf x}) \cong g(\mu_{\bf x}) + \nabla_{\bf x} g(\mu_{\bf x})^T 
({\bf x} - \mu_{\bf x}) + \frac{1}{2} ({\bf x} - \mu_{\bf x})^T 
\nabla^2_{\bf x} g(\mu_{\bf x}) ({\bf x} - \mu_{\bf x})
\label{eq:taylor2_x_mean}
\end{equation}

\item same as AMV/AMV$^2$, except that the Taylor series is expanded 
in u-space.  The first-order option has been termed the u-space AMV 
method:
\begin{equation}
G({\bf u}) \cong G(\mu_{\bf u}) + \nabla_u G(\mu_{\bf u})^T 
({\bf u} - \mu_{\bf u}) \label{eq:linear_u_mean}
\end{equation}
where $\mu_{\bf u} = T(\mu_{\bf x})$ and is nonzero in general, and 
the second-order option has been named the u-space AMV$^2$ method:
\begin{equation}
G({\bf u}) \cong G(\mu_{\bf u}) + \nabla_{\bf u} G(\mu_{\bf u})^T 
({\bf u} - \mu_{\bf u}) + \frac{1}{2} ({\bf u} - \mu_{\bf u})^T 
\nabla^2_{\bf u} G(\mu_{\bf u}) ({\bf u} - \mu_{\bf u}) 
\label{eq:taylor2_u_mean}
\end{equation}

\item an initial Taylor series approximation in x-space at the uncertain 
variable means, with iterative expansion updates at each MPP estimate
(${\bf x}^*$) until the MPP converges.  The first-order option is
commonly known as AMV+:
\begin{equation}
g({\bf x}) \cong g({\bf x}^*) + \nabla_x g({\bf x}^*)^T ({\bf x} - {\bf x}^*)
\label{eq:linear_x_mpp}
\end{equation}
and the second-order option has been named AMV$^2$+:
\begin{equation}
g({\bf x}) \cong g({\bf x}^*) + \nabla_{\bf x} g({\bf x}^*)^T 
({\bf x} - {\bf x}^*) + \frac{1}{2} ({\bf x} - {\bf x}^*)^T 
\nabla^2_{\bf x} g({\bf x}^*) ({\bf x} - {\bf x}^*) \label{eq:taylor2_x_mpp}
\end{equation}

\item same as AMV+/AMV$^2$+, except that the expansions are performed in 
u-space.  The first-order option has been termed the u-space AMV+ method.
\begin{equation}
G({\bf u}) \cong G({\bf u}^*) + \nabla_u G({\bf u}^*)^T ({\bf u} - {\bf u}^*)
\label{eq:linear_u_mpp}
\end{equation}
and the second-order option has been named the u-space AMV$^2$+ method:
\begin{equation}
G({\bf u}) \cong G({\bf u}^*) + \nabla_{\bf u} G({\bf u}^*)^T 
({\bf u} - {\bf u}^*) + \frac{1}{2} ({\bf u} - {\bf u}^*)^T 
\nabla^2_{\bf u} G({\bf u}^*) ({\bf u} - {\bf u}^*) \label{eq:taylor2_u_mpp}
\end{equation}

\item a multipoint approximation in x-space. This approach involves a 
Taylor series approximation in intermediate variables where the powers
used for the intermediate variables are selected to match information at
the current and previous expansion points.  Based on the 
two-point exponential approximation concept (TPEA, \cite{Fad90}), the 
two-point adaptive nonlinearity approximation (TANA-3, \cite{Xu98})
approximates the limit state as:
\begin{equation}
g({\bf x}) \cong g({\bf x}_2) + \sum_{i=1}^n 
\frac{\partial g}{\partial x_i}({\bf x}_2) \frac{x_{i,2}^{1-p_i}}{p_i} 
(x_i^{p_i} - x_{i,2}^{p_i}) + \frac{1}{2} \epsilon({\bf x}) \sum_{i=1}^n 
(x_i^{p_i} - x_{i,2}^{p_i})^2 \label{eq:tana_g}
\end{equation}

\noindent where $n$ is the number of uncertain variables and:
\begin{eqnarray}
p_i & = & 1 + \ln \left[ \frac{\frac{\partial g}{\partial x_i}({\bf x}_1)}
{\frac{\partial g}{\partial x_i}({\bf x}_2)} \right] \left/ 
\ln \left[ \frac{x_{i,1}}{x_{i,2}} \right] \right. \label{eq:tana_pi_x} \\
\epsilon({\bf x}) & = & \frac{H}{\sum_{i=1}^n (x_i^{p_i} - x_{i,1}^{p_i})^2 + 
\sum_{i=1}^n (x_i^{p_i} - x_{i,2}^{p_i})^2} \label{eq:tana_eps_x} \\
H & = & 2 \left[ g({\bf x}_1) - g({\bf x}_2) - \sum_{i=1}^n 
\frac{\partial g}{\partial x_i}({\bf x}_2) \frac{x_{i,2}^{1-p_i}}{p_i} 
(x_{i,1}^{p_i} - x_{i,2}^{p_i}) \right] \label{eq:tana_H_x}
\end{eqnarray}
\noindent and ${\bf x}_2$ and ${\bf x}_1$ are the current and previous
MPP estimates in x-space, respectively.  Prior to the availability of
two MPP estimates, x-space AMV+ is used.

\item a multipoint approximation in u-space. The u-space TANA-3
approximates the limit state as:
\begin{equation}
G({\bf u}) \cong G({\bf u}_2) + \sum_{i=1}^n 
\frac{\partial G}{\partial u_i}({\bf u}_2) \frac{u_{i,2}^{1-p_i}}{p_i} 
(u_i^{p_i} - u_{i,2}^{p_i}) + \frac{1}{2} \epsilon({\bf u}) \sum_{i=1}^n 
(u_i^{p_i} - u_{i,2}^{p_i})^2 \label{eq:tana_G}
\end{equation}

\noindent where:
\begin{eqnarray}
p_i & = & 1 + \ln \left[ \frac{\frac{\partial G}{\partial u_i}({\bf u}_1)}
{\frac{\partial G}{\partial u_i}({\bf u}_2)} \right] \left/ 
\ln \left[ \frac{u_{i,1}}{u_{i,2}} \right] \right. \label{eq:tana_pi_u} \\
\epsilon({\bf u}) & = & \frac{H}{\sum_{i=1}^n (u_i^{p_i} - u_{i,1}^{p_i})^2 + 
\sum_{i=1}^n (u_i^{p_i} - u_{i,2}^{p_i})^2} \label{eq:tana_eps_u} \\
H & = & 2 \left[ G({\bf u}_1) - G({\bf u}_2) - \sum_{i=1}^n 
\frac{\partial G}{\partial u_i}({\bf u}_2) \frac{u_{i,2}^{1-p_i}}{p_i} 
(u_{i,1}^{p_i} - u_{i,2}^{p_i}) \right] \label{eq:tana_H_u}
\end{eqnarray}
\noindent and ${\bf u}_2$ and ${\bf u}_1$ are the current and previous
MPP estimates in u-space, respectively.  Prior to the availability of
two MPP estimates, u-space AMV+ is used.

\item the MPP search on the original response functions without the 
use of any approximations.  Combining this option with first-order and
second-order integration approaches (see next section) results in the
traditional first-order and second-order reliability methods (FORM and
SORM).
\end{enumerate}

The Hessian matrices in AMV$^2$ and AMV$^2$+ may be available
analytically, estimated numerically, or approximated through
quasi-Newton updates.  The selection between x-space or u-space for
performing approximations depends on where the approximation will be
more accurate, since this will result in more accurate MPP estimates
(AMV, AMV$^2$) or faster convergence (AMV+, AMV$^2$+, TANA).  Since
this relative accuracy depends on the forms of the limit state $g(x)$
and the transformation $T(x)$ and is therefore application dependent
in general, DAKOTA supports both options.  A concern with
approximation-based iterative search methods (i.e., AMV+, AMV$^2$+ and
TANA) is the robustness of their convergence to the MPP.  It is
possible for the MPP iterates to oscillate or even diverge.  However,
to date, this occurrence has been relatively rare, and DAKOTA contains
checks that monitor for this behavior.  Another concern with TANA is
numerical safeguarding (e.g., the possibility of raising negative
$x_i$ or $u_i$ values to nonintegral $p_i$ exponents in
Equations~\ref{eq:tana_g}, \ref{eq:tana_eps_x}-\ref{eq:tana_G},
and~\ref{eq:tana_eps_u}-\ref{eq:tana_H_u}).  Safeguarding involves
offseting negative $x_i$ or $u_i$ and, for potential numerical
difficulties with the logarithm ratios in Equations~\ref{eq:tana_pi_x}
and~\ref{eq:tana_pi_u}, reverting to either the linear ($p_i = 1$) or
reciprocal ($p_i = -1$) approximation based on which approximation has
lower error in $\frac{\partial g}{\partial x_i}({\bf x}_1)$ or
$\frac{\partial G}{\partial u_i}({\bf u}_1)$.

\subsubsection{Probability integrations} \label{uq:reliability:local:mpp:int}

The second algorithmic variation involves the integration approach for
computing probabilities at the MPP, which can be selected to be
first-order (Equations~\ref{eq:p_cdf}-\ref{eq:p_ccdf}) or second-order
integration.  Second-order integration involves applying a curvature
correction~\cite{Bre84,Hoh88,Hon99}.  Breitung applies a correction
based on asymptotic analysis~\cite{Bre84}:
\begin{equation}
p = \Phi(-\beta_p) \prod_{i=1}^{n-1} \frac{1}{\sqrt{1 + \beta_p \kappa_i}}
\label{eq:p_2nd_breit}
\end{equation}
where $\kappa_i$ are the principal curvatures of the limit state
function (the eigenvalues of an orthonormal transformation of
$\nabla^2_{\bf u} G$, taken positive for a convex limit state) and
$\beta_p \ge 0$ (a CDF or CCDF probability correction is selected to
obtain the correct sign for $\beta_p$).  An alternate correction in
\cite{Hoh88} is consistent in the asymptotic regime ($\beta_p \to \infty$) 
but does not collapse to first-order integration for $\beta_p = 0$:
\begin{equation}
p = \Phi(-\beta_p) \prod_{i=1}^{n-1} 
\frac{1}{\sqrt{1 + \psi(-\beta_p) \kappa_i}} \label{eq:p_2nd_hr}
\end{equation}
where $\psi() = \frac{\phi()}{\Phi()}$ and $\phi()$ is the standard
normal density function.  \cite{Hon99} applies further corrections to
Equation~\ref{eq:p_2nd_hr} based on point concentration methods.  At
this time, all three approaches are available within the code, but the
Hohenbichler-Rackwitz correction is used by default (switching the 
correction is a compile-time option in the source code and has not
not currently been exposed in the input specification).


\section{Global Reliability Methods}\label{uq:reliability:global}

Global reliability methods include the efficient global reliability
analysis (EGRA) method.

% TO DO: add global reliability description from AIAA paper(s)
