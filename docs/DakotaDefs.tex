% Shortcuts for DAKOTA information incorporated in multiple versions of manuals
%
% This file is includeded in all top-level .tex files

% When using this command, likely follow with a forced space, e.g., one of
% Version \DakotaVersion\ Manual
% Version \DakotaVersion\space Manual
% Version \DakotaVersion{} Manual
%%\newcommand{\DakotaVersion}{5.1}
\newcommand{\DakotaVersion}{5.1+}


% Most recent SAND report numbers and report date based on
% Data should be SAND report date and not reflect updates
\newcommand{\DakotaSANDDate}{December 2009}
\newcommand{\DakotaSANDDev}{SAND2010-2185}
\newcommand{\DakotaSANDRef}{SAND2010-2184}
\newcommand{\DakotaSANDUsers}{SAND2010-2183}
\newcommand{\DakotaSANDTheory}{SAND2011-xxxx}


% Need to manually break line or Eldred gets incorrectly split
\newcommand{\DakotaAuthorSAND}{
Brian~M.~Adams, William~J.~Bohnhoff, Keith~R.~Dalbey, John~P.~Eddy,\\
Michael~S.~Eldred, David~M.~Gay, Karen~Haskell, Patricia~D.~Hough,\\
Sophia~Lefantzi, Laura~P.~Swiler
}

\newcommand{\DakotaAuthorLong}{
Brian~M.~Adams, Keith~R.~Dalbey, Michael~S.~Eldred, David~M.~Gay, 
Laura~P.~Swiler\\
Optimization and Uncertainty Quantification Department\\
\\
William~J.~Bohnhoff\\
Radiation Transport Department\\
\\
John~P.~Eddy\\
System Readiness and Sustainment Technologies Department\\
\\
Karen~Haskell\\
Scientific Applications and User Support Department\\
\\
Sandia National Laboratories\\
P.O. Box 5800\\
Albuquerque, NM 87185\\
\\
Patricia~D.~Hough, Sophia~Lefantzi\\
Quantitative Modeling and Analysis Department\\
\\
Sandia National Laboratories\\
P.O. Box 969\\
Livermore, CA 94551
}

\newcommand{\DakotaAuthorFormatted}{
{\large \bf Brian M. Adams, Keith R. Dalbey, Michael S. Eldred, David
M. Gay, Laura P. Swiler}\\ 
{\large Optimization and Uncertainty Quantification Department}\\
\vspace*{0.5cm}
{\large \bf William J. Bohnhoff}\\
{\large Radiation Transport Department}\\
\vspace*{0.5cm}
{\large \bf John P. Eddy}\\
{\large System Readiness and Sustainment Technologies Department}\\
\vspace*{0.5cm}
{\large \bf Karen Haskell}\\
{\large Scientific Applications and User Support Department}\\
\vspace*{0.5cm}
{\large Sandia National Laboratories}\\
{\large P.O. Box 5800}\\
{\large Albuquerque, New Mexico 87185}\\
\vspace*{1cm}
{\large \bf Patricia D. Hough, Sophia Lefantzi}\\
{\large Quantitative Modeling and Analysis Department}\\
\vspace*{0.5cm}
{\large Sandia National Laboratories}\\
{\large P.O. Box 969}\\
{\large Livermore, CA 94551}\\
}


% Fragments which together comprise abstracts

\newcommand{\DakotaAbstractShared}{
The DAKOTA (Design Analysis Kit for Optimization and Terascale
Applications) toolkit provides a flexible and extensible interface
between simulation codes and iterative analysis methods. DAKOTA
contains algorithms for optimization with gradient and
nongradient-based methods; uncertainty quantification with sampling,
reliability, and stochastic expansion methods; parameter
estimation with nonlinear least squares methods; and
sensitivity/variance analysis with design of experiments and parameter
study methods. These capabilities may be used on their own or as
components within advanced strategies such as surrogate-based
optimization, mixed integer nonlinear programming, or optimization
under uncertainty. By employing object-oriented design to implement
abstractions of the key components required for iterative systems
analyses, the DAKOTA toolkit provides a flexible and extensible
problem-solving environment for design and performance analysis of
computational models on high performance computers.
% blank line intended

% blank line intended
}

\newcommand{\DakotaAbstractDev}{
This report serves as a developers manual for the DAKOTA software and
describes the DAKOTA class hierarchies and their interrelationships.
It derives directly from annotation of the source code and provides
detailed class documentation, including all member functions and
attributes.  
}

\newcommand{\DakotaAbstractRef}{
This report serves as a reference manual for the commands specification
for the DAKOTA software, providing input overviews, option descriptions,
and example specifications.
}

\newcommand{\DakotaAbstractUsers}{
This report serves as a user's manual for the DAKOTA software and
provides capability overviews and procedures for software execution,
as well as a variety of example studies.
}

\newcommand{\DakotaAbstractTheory}{
This report serves as a theoretical manual for selected algorithms
implemented within the DAKOTA software.  It is not intended as a
comprehensive theoretical treatment, since a number of existing texts
cover general optimization theory, statistical analysis, and other
introductory topics.  Rather, this manual is intended to summarize a
set of DAKOTA-related research publications in the areas of
surrogate-based optimization, uncertainty quantification, and
optimization under uncertainty that provide the foundation for many 
of DAKOTA's iterative analysis capabilities.
}
